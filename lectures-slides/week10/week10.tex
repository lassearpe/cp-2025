\PassOptionsToPackage{colorlinks,linkcolor={blue},citecolor={blue},urlcolor={blue},breaklinks=true,final}{hyperref}
\PassOptionsToPackage{dvipsnames}{xcolor}
\documentclass[xcolor={dvipsnames,svgnames},aspectratio=169]{beamer}

\usepackage{fontawesome5}
\usepackage{booktabs} % For better table formatting

\title{Concurrency Programming}
\subtitle{Week 10 (Lecture 2)}
\author{Stelios Tsampas}
\institute{
  \faEnvelope \; stelios@imada.sdu.dk
  \qquad
  \faGlobe \;
  \href{https://www.steliostsampas.com}{https://www.steliostsampas.com}
  \\\\\
  \faGithub \; stelios-tau/cp-2025
  \qquad\;\;
    \faDiscord \; cp-2025
}
\date{February 17, 2025}

\titlegraphic{\includegraphics[height=0.6cm,keepaspectratio]{../media/sdu-black.eps}}

\usetheme[block=fill]{metropolis}


%\usepackage{pres-common}
\usepackage{textpos}
\usepackage{centernot}

% \newcommand{\Goesv}[3]{\ensuremath{#1 \xRightarrow{~#3~} #2}}
% \newcommand{\goesv}[3]{\ensuremath{#1 \xrightarrow{~#3~} #2}}

% \usepackage{etex}
% \usepackage{semantic}

\usepackage[utf8]{inputenc}
\usepackage[english]{babel}
\usepackage{tikz}
\usepackage{hyperref}

\usetikzlibrary{arrows,shapes,matrix}
\usetikzlibrary{backgrounds}
\usetikzlibrary{positioning}
\usetikzlibrary{automata}
\usetikzlibrary{mindmap}
\usetikzlibrary{shapes.callouts}
\usetikzlibrary{decorations.text}
\usetikzlibrary{tikzmark}
\usetikzlibrary{calc}
\usetikzlibrary{overlay-beamer-styles}

\tikzset{onslide/.code args={<#1>#2}{%
    \only<#1>{\pgfkeysalso{#2}} % \pgfkeysalso doesn't change the path
  }}

\setbeamercolor{mygray}{bg=Gray!20}

\tikzset{temporal/.code args={<#1>#2#3#4}{%
    \temporal<#1>{\pgfkeysalso{#2}}{\pgfkeysalso{#3}}{\pgfkeysalso{#4}} % \pgfkeysalso doesn't change the path
  }}

\tikzstyle{highlight}=[fill=green!50]
\tikzstyle{hgreen}=[fill=green!20]
\tikzstyle{hred}=[fill=red!50]
\tikzstyle{hblue}=[fill=blue!50]
\tikzstyle{hgray}=[fill=gray!50]

\addtobeamertemplate{frametitle}{}{%
\begin{textblock*}{100mm}(\textwidth-2cm,-0.86cm)
\includegraphics[height=0.6cm,keepaspectratio]{../media/sdu-white.eps}
\end{textblock*}}

%\usepackage{tikz-cd}
% \usepackage{wasysym}
% \usepackage{color}
% \usepackage[matrix,arrow]{xy}
% \xyoption{all}
% \SelectTips{cm}{}
% % \usepackage{cite}
% \usepackage{amsthm}
% \usepackage{amsmath}
% \usepackage{bbold}
% % \usepackage[bbgreekl]{mathbbol}
% \usepackage{amssymb}
% \usepackage{pifont}
% \usepackage{mathtools}
% \usepackage{amsbsy}
% % \usepackage{paralist}
% \usepackage{shadethm}
% % \usepackage{fancyhdr}
% \usepackage{stmaryrd}
% \usepackage{wasysym}
% \usepackage{graphicx}
% \usepackage{tabularx}
% \usepackage{dsfont}
% \usepackage{ulem}




%\bibliography{mainBiblio}

%\includeonlyframes{current}
\begin{document}

\frame{\titlepage}

\def\firstcircle{(0,0) circle (2cm)}
\def\secondcircle{(1.4,1.4) circle (2cm)}
\def\thirdcircle{(0:2.4) circle (2cm)}

\begin{frame}[fragile]
  \frametitle{Examples of Concurrency}

  Picture a chef (or more than one) cooking multiple dishes at the same time:

  \begin{itemize}
  \item[\faBook]<1-> The chef prepares multiple meals at once by switching tasks.
  \item[\faBook]<2-> They might chop vegetables, then stir a sauce, then check the oven.
  \item[\faBook]<3-> The dishes are interleaved but not truly parallel.
  \end{itemize}
\end{frame}

\begin{frame}[fragile]
  \frametitle{Case 1: one chef}

  \begin{center}
    \includegraphics[width=11cm,keepaspectratio]{../media/lecture1-chef.png}
  \end{center}

  \onslide<2->{
    \begin{tikzpicture}[overlay, remember picture]
      \node[xshift=3cm,yshift=5cm,starburst,starburst points=30,
      align=center,fill=yellow, opacity=1,draw=red, line width=2pt]
      {\textbf{Nondeterministic \faSkullCrossbones}};
    \end{tikzpicture}}

\end{frame}

% \begin{frame}[fragile]
%   \frametitle{Case 1: one chef}

%   \large{
%   \begin{itemize}
%   \item[\faBook]<1-> You don't need more than one chef/CPU cores/threads to have
%     concurrent execution.
%   \item[\faBook]<1-> \textbf{Task switching}/\textbf{interleaving} are the hallmarks of
%     concurrency.
%   \end{itemize}}
% \end{frame}

\begin{frame}[fragile]
  \frametitle{Degenerate (non-)example, sequential execution}

  \begin{center}
    \includegraphics[width=11cm,keepaspectratio]{../media/lecture1-seq.png}
  \end{center}

  \onslide<2->{
    \begin{tikzpicture}[overlay, remember picture]
      \node[xshift=3cm,yshift=5cm,ellipse,
      align=center,fill=green!20, opacity=1,draw=black, line width=0.4pt]
      {\textbf{Predictable \faCheck}};
    \end{tikzpicture}}

\end{frame}

% \begin{frame}[fragile]
%   \frametitle{Degenerate (non-)example, sequential execution}

%   \large{
%     \begin{itemize}
%     \item[\faBook]<1-> In sequential execution, tasks are \emph{sequenced},
%       their execution will not be interrupted.
%     \item[\faBook]<2-> Simple and predictable, the kind of programming you have
%       been doing so far.
%     \end{itemize}}

% \end{frame}


\begin{frame}[fragile]
  \frametitle{Case 2: two chefs}

  \begin{center}
    \includegraphics[width=11cm,keepaspectratio]{../media/lecture1-twochefs.png}
  \end{center}

  \onslide<2->{
    \begin{tikzpicture}[overlay, remember picture]
      \node[xshift=3cm,yshift=5cm,starburst,starburst points=30,
      align=center,fill=yellow, opacity=1,draw=red, line width=2pt]
      {\textbf{Nondeterministic \faSkullCrossbones}};
    \end{tikzpicture}}
\end{frame}

% \begin{frame}[fragile]
%   \frametitle{Case 2: two chefs}

%   \large{
%     \begin{itemize}
%     \item[\faBook]<1-> This time there are actually distinct chefs/\textbf{threads} of
%       execution taking place.
%     \item[\faBook]<2-> They might be sharing resources (i.e. using the same
%       equipment, food), so extra care needs to be taken, especially because the
%       interleaving is \textbf{non-deterministic}.
%     \item[\faBook]<3-> Remember that these \textbf{threads} are abstractions, it
%       is not necessary that, underneath the surface, their a multi-core CPU
%       executing each thread.
%     \end{itemize}}
% \end{frame}

\begin{frame}[fragile]
  \frametitle{Degenerate (non-)example 2, parallelism}

  \begin{center}
    \includegraphics[width=11cm,keepaspectratio]{../media/lecture1-para.png}
  \end{center}

  \onslide<2->{
    \begin{tikzpicture}[overlay, remember picture]
      \node[xshift=3cm,yshift=5cm,ellipse,
      align=center,fill=green!20, opacity=1,draw=black, line width=0.4pt]
      {\textbf{Predictable \faCheck}};
    \end{tikzpicture}}
\end{frame}

% \begin{frame}[fragile]
%   \frametitle{Degenerate (non-)example 2, parallelism}

%   \large{
%     \begin{itemize}
%     \item[\faBook]<1-> Again, there are two chefs/\textbf{threads} of
%       execution taking place.
%     \item[\faBook]<1-> However, their tasks are distinct, and there is no
%       communication between the two.
%     \item[\faBook]<1-> This is \textbf{parallel programming}, a different topic
%       that is about the developing algorithms that make use of
%       multiple CPU units.
%     \item[\faBook]<2-> \textbf{Not} quite the topic of this class,
%       although the programming techniques involved may overlap.
%     \end{itemize}}

% \end{frame}

% \begin{frame}{Key Differences: Concurrency vs. Parallelism}
%   \vspace{-0.4cm}
% \begin{table}[]
%     \centering
%     \begin{tabular}{l|l|l}
%         \toprule
%       \textbf{Feature}
%       & \textbf{Concurrency}
%       & \textbf{Parallelism} \\
%         \midrule
%       \textbf{Definition}
%       & Interleaved execution
%       & True simultaneous execution\\
%         \midrule
%       \textbf{Execution}
%       & Task switching, may be single-core
%       & Requires multiple cores \\
%         \midrule
%       \textbf{Use Case}
%       & Responsiveness (UI, servers)
%       & Performance (big data, heavy comp.) \\
%         \midrule
%       \textbf{Hardware}
%       & Can work on single-core
%       & Requires multi-core CPU \\
%         \midrule
%       \textbf{Example}
%       & Multi-threading with task switching
%       & Multi-threading with multiple cores \\
%         \midrule
%       \textbf{Java}
%       & \texttt{Thread}, \texttt{ExecutorService}
%       & \texttt{ForkJoinPool}, \texttt{parallelStream()} \\
%       \midrule
%       \textbf{Resources}
%       & Shared resources, synchronization
%       & Little to no sharing and communication\\
%       \midrule
%       \textbf{Effect}
%       & Non-deterministic
%       & (Largely) deterministic\\
%         \bottomrule
%     \end{tabular}
%   \end{table}
%   \onslide<3->{
%     \begin{tikzpicture}[overlay, remember picture]
%       \node[xshift=4cm,yshift=3cm,starburst,starburst points=25,
%       align=center,fill=yellow, opacity=1,draw=red, line width=2pt]
%       {\large{\textbf{Complete chaos!}}};
%     \end{tikzpicture}
%     \begin{tikzpicture}[overlay, remember picture]
%       \node[xshift=11cm,yshift=3cm,ellipse,
%       align=center,fill=green!20, opacity=1,draw=black, line width=0.4pt]
%       {\large{\textbf{Predictable \faCheck}}};
%     \end{tikzpicture}}
% \end{frame}

\begin{frame}[fragile]
  \frametitle{What this is all about}

  \large{A large amount of this class is about managing the \textbf{sharing} of resources
  and generally the \textbf{communication} and \textbf{coordination} of these
  \textbf{threads} of execution. \uncover<2->{Yes, the term \emph{threads} is
    the technically correct term.}}
\end{frame}

\begin{frame}[fragile]
  \frametitle{Threads}

  \begin{itemize}
  \item[\faBook] The main abstraction that is involved in concurrency.
  \item[\faBook] A thread represents a lightweight, independent unit of
    execution.
  \item[\faBook] Threads are present at multiple levels of the
    software stack:
    \begin{itemize}
    \item[\faLinux] Operating system level, i.e. user and kernel threads.
    \item[\faCode] At the level of programming languages, Java threads, pthreads
      in C...
    \end{itemize}
  \item[\faBook] There can be many active threads active at the same time, i.e.
    \emph{concurrently}.
  \item[\faExclamation]<2-> ... but they do not need to run at the actual same moment.
  \end{itemize}
\end{frame}

\begin{frame}[fragile]
  \frametitle{Threads vs processes}

  \begin{center}
    \includegraphics[width=11cm,keepaspectratio]{../media/lecture1-threads.png}
  \end{center}
\end{frame}

\section{(Proper) Introduction to Threads}

\begin{frame}[fragile]
  \frametitle{A recipe}

  Suppose that two steps of the recipe for \textcolor{blue}{Dish I} read as
  \begin{enumerate}
    \item Put veggies in the (heated) frying pan. Wait 8 minutes. Take veggies
      out.
    \item Put meat in the (heated) frying pan. Wait 8 minutes. Take meat out.
    \end{enumerate}
    \vspace{0.6cm}
    \begin{block}{Question 1}
      \uncover<2->{Is everything alright with this recipe?}
    \end{block}
    \begin{block}{Question 2}
      \uncover<3->{What if two chefs (i.e. two threads) are working on the recipe?}
    \end{block}
\end{frame}

\begin{frame}[fragile]
  \frametitle{What is thread safety?}

  \begin{block}{Thread-safe class}
    A class (or piece of code) is \emph{thread-safe} if it behaves correctly
    when accessed from multiple threads, regardless of the scheduling or
    interleaving of the execution of those threads by the runtime environment,
    and with no additional synchronization or other coordination on the part of
    the calling code.
  \end{block}

  \vspace{1cm}
  \begin{block}{Example}
       \uncover<2->{\emph{The \textcolor{blue}{Dish I} \textbf{recipe} is not
           thread-safe.}}
  \end{block}
\end{frame}

\begin{frame}[fragile]
  \large{\textbf{Important:} There is not pre-existing notion of correctness.
    The developer decides what constitutes what is correct and what is not.}
\end{frame}

\begin{frame}[fragile]
  \frametitle{What is a race condition}

  \begin{block}{Race condition}
    A race condition occurs when the correctness of a computation depends
    on the relative timing or interleaving of multiple threads by the runtime.
  \end{block}

  \begin{itemize}
  \item[\faBook] More specific than thread-safety.
  \end{itemize}

    \begin{block}{Example}
      \uncover<2->{\emph{The correct completion of the recipes by two chefs
          depends on the timing of the usage of the frying pan. In other words,
          we have a \textcolor{red}{single} \textbf{race condition} in the
          ``2-chefs-2-dishes-kitchen'' program.}}
  \end{block}
\end{frame}

\begin{frame}[fragile]
  \frametitle{Key idea in concurrency}

  Sharing of mutable state (e.g. the frying pan).
\end{frame}

% \begin{frame}[fragile]
%   \frametitle{Dangerous pitfalls (will be dealing a lot with those)}

%   \begin{itemize}
%   \item[\faUserInjured] \textbf{Race conditions}: Multiple threads accessing and
%     modifying shared data inconsistently.
%     \begin{itemize}
%     \item[\faBriefcaseMedical]<2-> Proper, careful synchronization.
%     \end{itemize}
%   \item[\faUserInjured] \textbf{Deadlocks}: Threads waiting indefinitely for each
%     other.
%     \begin{itemize}
%     \item[\faBriefcaseMedical]<2-> Proper, careful synchronization.
%     \end{itemize}
%   \item[\faUserInjured] \textbf{Starvation}: Some threads never get CPU time due to
%     scheduling.
%     \begin{itemize}
%     \item[\faBriefcaseMedical]<2-> Proper, careful synchronization.
%     \end{itemize}
%   \item[\faUserInjured] \textbf{Visibility Issues}: Changes in one thread
%     not visible to others.
%     \begin{itemize}
%     \item[\faBriefcaseMedical]<2-> Make no assumptions on data sharing between threads.
%     \end{itemize}
%   \end{itemize}
% \end{frame}

% \section{Course Schedule}

% \begin{frame}[allowframebreaks]
%   \frametitle{Bibliography}
%   \printbibliography
% \end{frame}

\end{document}

%%% Local Variables:
%%% mode: latex
%%% TeX-engine: xetex
%%% TeX-master: t
%%% End:
